\label{ch:unfolding_methods}

%\section{Theory of Neutron Noise Source Unfolding}
%\label{sec:unfolding_theory}
%Neutron noise source unfolding is a method that is used to identify the type of noise source and locate the position at which the perturbations occur. Generally, the noise source unfolding method primarily depends on Green’s function. Hence, determining Green’s function from the neutron noise equation is necessary \cite{demaziereIdentificationLocalizationAbsorbers2005}. The following method uses Green’s function for the neutron noise unfolding.
%
%\subsection{Inversion Method}
%
%As the name suggests, the inversion method inverts Green’s function to estimate the noise sources in the system. Recalling the definition of flux perturbations:
%\begin{equation}
%    \delta \phi (\textbf{r},\omega) = \int_{V_p} G(\textbf{r}, \textbf{r}_p, \omega) S(\textbf{r}_p, \omega) dV_p
%    \label{eq:inversion_definition}
%\end{equation}
%In matrix form, Equation \ref{eq:inversion_definition} can be written as matrix-vector multiplication:
%\begin{equation}
%    \delta \phi (\textbf{r},\omega) = G(\textbf{r}_p \rightarrow \textbf{r}, \omega) S(\textbf{r}_p, \omega)
%    \label{eq:inversion_definition_matrix}
%\end{equation}
%Therefore, the noise source can be determined by inverting the Green’s function such that:
%\begin{equation}
%    S(\textbf{r}_p, \omega) = G^{-1}(\textbf{r}_p \rightarrow \textbf{r}, \omega) \delta \phi (\textbf{r},\omega) 
%    \label{eq:source_inversion_matrix}
%\end{equation}
%
%Equation \ref{eq:inversion_definition_matrix} is trivial to solve if Green’s function and the flux perturbations at each computational node are known. However, in practice, only a few detectors can be used to measure the neutron flux. Thus, only a few elements of Green’s function and the flux perturbation can be obtained from the detectors. 
%
%An alternative proposed by \cite{demaziereIdentificationLocalizationAbsorbers2005} is to interpolate the detector readings to match the vector size $\delta \phi (\textbf{r},\omega)$. In the paper, \cite{demaziereIdentificationLocalizationAbsorbers2005} assumes that the detector is sensitive to thermal neutron and the noise source is known to be a perturbation of thermal absorption cross-section. Therefore, the interpolation is written as:
%\begin{equation}
%    S(\textbf{r}_p, \omega) = G^{-1}(\textbf{r}_p \rightarrow \textbf{r}_{\text{interp}}, \omega) \delta \phi (\textbf{r}_{\text{interp}},\omega) 
%    \label{eq:source_inversion_matrix2}
%\end{equation}
%where $G^{-1}(\textbf{r}_p \rightarrow \textbf{r}_{\text{interp}}$ has size $N \times N$, $\delta \phi (\textbf{r}_{\text{interp}},\omega)$ has size $N$, and $S(\textbf{r}_p, \omega)$ has size $N$.
%
%Using an interpolation method, \cite{demaziereIdentificationLocalizationAbsorbers2005} were able to reconstruct the noise source using an interpolation method. However, the inversion might have some problems in the boundaries because the interpolated thermal flux perturbation is forced to be equal to zero outside the reflector nodes. Even though the reconstruction at the boundaries produced poor approximation, the overall process is still providing promising results. The result of this method is highly dependent on the interpolation process. Incorrect interpolation would lead to inaccuracies in the unfolding.
%
%\subsection{Zoning Method}
%
%The zoning method is an extension of the inversion method. The difference between the two lies in using zones in the zoning method instead of interpolation. If one assumes that the system is divided into several zones $Z_k$, where each of the zones has the same number of detectors, then the flux perturbation can be written as \cite{demaziereIdentificationLocalizationAbsorbers2005}:
%
%\begin{equation}
%    \delta \phi (\textbf{r}_{\text{meas}},\omega) = \sum_{k} G(\textbf{r}_{Z_k} \rightarrow \textbf{r}_{\text{meas}}, \omega) S(\textbf{r}_{Z_k}, \omega)
%    \label{eq:zoning_definition}
%\end{equation}
%Note that all of the $G(\textbf{r}_{Z_k} \rightarrow \textbf{r}_{\text{meas}}, \omega)$ are square matrices since $\delta \phi (\textbf{r}_{\text{meas}},\omega)$ and $S(\textbf{r}_{Z_k}, \omega)$ have the same size (number of detectors).
%
%If the fuel assemblies represented in the zone $Z_k$ are set not to be close to each other, the response of the thermal flux detector will be unique to each other. Having the fuel assemblies belonging to a given zone $Z_k$ evenly distributed throughout the core is an easy and practical way to achieve such a goal.
%
%Then, by inverting one of the matrices $G(\textbf{r}_{Z_k} \rightarrow \textbf{r}_{\text{meas}}, \omega)$ for zone $Z_l$, one can obtain:
%\begin{equation}
%    G^{-1}(\textbf{r}_{Z_l} \rightarrow \textbf{r}_{\text{meas}}, \omega) \times \delta \phi (\textbf{r}_{\text{meas}},\omega) = G^{-1}(\textbf{r}_{Z_l} \rightarrow \textbf{r}_{\text{meas}}, \omega) \biggl(\sum_{k} G(\textbf{r}_{Z_k} \rightarrow \textbf{r}_{\text{meas}}, \omega) S(\textbf{r}_{Z_k}, \omega) \biggr) + S(\textbf{r}_{Z_l}, \omega)
%    \label{eq:zoning_definition_2}
%\end{equation}
%Moving forward, we will assume that the noise source is located at $Z_s$, which might be at one of the $Z_k$ or at $Z_l$. We have two possible cases:
%\begin{enumerate}
%    \item If the noise source is located at the zone $ Z_l=Z_s$, then one can write:
%    \begin{equation}
%        G^{-1}(\textbf{r}_{Z_l} \rightarrow \textbf{r}_{\text{meas}}, \omega) \times \delta \phi (\textbf{r}_{\text{meas}},\omega) = S(\textbf{r}_{Z_l}, \omega)
%        \label{eq:zoning_case1}
%    \end{equation}
%    In this case, $S(\textbf{r}_{Z_l}) = S(\textbf{r}_{Z_s})$ is the “true” noise source vector. The noise source vector would have a distinct peak from any other zone, indicating that the noise source is in the zone $Z_l$. 
%
%    \item If the inversion is done to the matrix corresponding to zone $Z_l \neq Z_s$ (contains no noise source), then Equation \ref{eq:zoning_definition_2} becomes:
%    \begin{equation}
%        G^{-1}(\textbf{r}_{Z_l} \rightarrow \textbf{r}_{\text{meas}}, \omega) \times \delta \phi (\textbf{r}_{\text{meas}},\omega) = G^{-1}(\textbf{r}_{Z_l} \rightarrow \textbf{r}_{\text{meas}}, \omega) G(\textbf{r}_{Z_k} \rightarrow \textbf{r}_{\text{meas}}, \omega) S(\textbf{r}_{Z_k}, \omega)        
%        \label{eq:zoning_case2}
%    \end{equation}
%    This means that the zone $Z_l$ does not contain the noise source vector. The right-hand side of the equation will be relatively flat and no peak will be visible \cite{demaziereIdentificationLocalizationAbsorbers2005}.
%\end{enumerate}
%
%To simplify the analysis, we can perform the following calculation.
%\begin{equation}
%    G^{-1}(\textbf{r}_{Z_l} \rightarrow \textbf{r}_{\text{meas}}, \omega) \times \delta \phi (\textbf{r}_{\text{meas}},\omega) = S_{\text{fict}}(\textbf{r}_{Z_l}, \omega)
%    \label{eq:zoning_simple}
%\end{equation}
%where $Z_l \in Z_k$. This calculation is done for every zone, which results in $k$ number of $S_{\text{fict}}(\textbf{r}_{Z_l}, \omega)$. For each $S_{\text{fict}}(\textbf{r}_{Z_l}, \omega)$, it should return to one of the either cases. If the zone $Z_l$ If the signal contains a noise source, then one of the vector elements should be significantly higher than the others. On the other hand, if a zone $Z_l$ does not contain a noise source, the elements of $S_{\text{fict}}(\textbf{r}_{Z_l}, \omega)$ would not be different from each other.
%
%The zoning method generally improves the accuracy of the noise unfolding. Some errors were detected when the noise source was located near the boundary of several detectors. In this location, the detector response will not be unique to each other, making the Green’s function inaccurate \cite{demaziereIdentificationLocalizationAbsorbers2005}.
%
%\subsection{Scanning method}
%
%The third method is the scanning method. This method compares the perturbations detected by the detector with the calculated response for all possible locations of the noise source within the system. The noise source is accurately identified when the calculated neutron noise matches the measured neutron noise \cite{demaziereIdentificationLocalizationAbsorbers2005}. This method was developed by \cite{karlssonLocalizationChannelInstability1999}, and extended by \cite{demaziereDevelopmentNoisebasedMethod2002}. Both investigations successfully determined the location of the unseated fuel assembly in the Swedish Forsmark-1 BWR. Both use Green’s function as the means to compare detector readings.
%
%This method starts with the definition of noise flux at a given location r induced by a noise source located at the location $\textbf{r}_p$ as follows:
%\begin{equation}
%    \delta \phi (\textbf{r}_i,\omega) = G(\textbf{r}_{p,j} \rightarrow \textbf{r}_i, \omega) S(\textbf{r}_{p,j}, \omega)
%    \label{eq:scanning_definition_matrix}
%\end{equation}
%Using Equation \ref{eq:scanning_definition_matrix} to compare with the detector reading will be difficult since the location and strength of the noise are unknown. If one has access to two detectors at locations A and B, the ratio between the neutron noise at these two locations can be used to eliminate the noise source strength:
%\begin{equation}
%    \frac{\delta \phi (\textbf{r}_A,\omega)}{\delta \phi (\textbf{r}_B,\omega)} = \frac{G(\textbf{r}_{p,j} \rightarrow \textbf{r}_A, \omega)}{G(\textbf{r}_{p,j} \rightarrow \textbf{r}_B, \omega)}
%    \label{eq:scanning_definition_frac}
%\end{equation}
%
%Then, the scanning algorithm consists of minimizing the following,
%\begin{equation}
%    \Delta(\textbf{r}) = \mathlarger{\mathlarger{\sum}}_{A,B} \biggl| \frac{\delta \phi (\textbf{r}_A,\omega)}{\delta \phi (\textbf{r}_B,\omega)} - \frac{G(\textbf{r}_{p,j} \rightarrow \textbf{r}_A, \omega)}{G(\textbf{r}_{p,j} \rightarrow \textbf{r}_B, \omega)} \biggr|
%    \label{eq:scanning_definition_min}
%\end{equation}
%Equation \ref{eq:scanning_definition_min} implies that the summation is done for all detector pairs for all locations $\textbf{r}$. If there exist $n$ number of detectors in the reactor, thus the number of detector pairs are as follows:
%\begin{equation}
%    N_{\text{detector pair}} = \frac{n \times (n+1)}{2}
%\end{equation}
%The location at which the minimum value of $\Delta(\textbf{r})$ is found indicates the location of the noise source. The magnitude of the noise source can be calculated as follows:
%\begin{equation}
%    S(\textbf{r}_p,\omega) = \frac{\delta \phi(r_m,\omega)}{G(\textbf{r}_{p} \rightarrow r_m, \omega)}
%\end{equation}
%for any detector m.
%
%The results from \cite{demaziereDevelopmentNoisebasedMethod2002} show that the scanning algorithm was able to locate any noise source correctly, provided there is no background. The drawback of this method is that it requires more computational power to compare every possible location of the noise source with every combination of detectors used for the evaluation.
%
%
%
%\section{Application of the Unfolding Method to HTTR Benchmark}
%\label{sec:httr_unfolding}
%\subsection{Problem Description}
%
%\subsection{Inversion Method}
%
%\subsection{Zoning Method}
%
%\subsection{Scanning Method}
%