\label{ch:simulations}

The neutron 
\section{Spatial Discretization for Rectangular Geometries}
\label{sec:rectangular_discretization}
For the rectangular geometries, the system is divided into uniform meshes using rectilinear grid. The size of the grid is an input for the user to adjust. As seen in Figure \ref{fig:rect_discretize}, the discretization is uniform with $dx$ and $dy$ defined by the user. 
\begin{figure}[h]
        \centering
        \includegraphics[width=0.6\textwidth]{figures/rect_discretize.png}
        \caption{2D rectangular discretization}
        \label{fig:rect_discretize}
\end{figure}

Equation \ref{eq:freq_4} are discretized and averaged over each mesh of the grid using box scheme finite difference. The box scheme finite difference is one of the finite difference method that uses a box-shaped control volume for discretization. The method is using center point of the mesh as reference of the discretization. The method still relies on Taylor expansion to obtain the discretization of the derivatives. To obtain the expression of the discretized neutron noise equation, we integrate Equation \ref{eq:freq_4} over $x \in [x_{i - \frac{1}{2}}, x_{i + \frac{1}{2}}]$ and $y \in [y_{i - \frac{1}{2}}, y_{i + \frac{1}{2}}]$.
The integration results for generic reaction rates can defined as follow:
\begin{equation}
        \int_{x_{i - \frac{1}{2}}}^{x_{i + \frac{1}{2}}} \int_{y_{i - \frac{1}{2}}}^{y_{i + \frac{1}{2}}} \Sigma_{g}(\textbf{r}, \omega) \phi_{g}(\textbf{r}, \omega) dx dy = \Sigma_{g,j,i}(\omega) \phi_{g,j,i}(\omega) \Delta x \Delta y
\end{equation}
where $g$ is index for neutron energy group and $\Delta x \Delta y$ is the size of the mesh $n$. The discretization of diffusion term depends on which axis of interest. The diffusion term in x-direction is defined as follows.
\begin{equation}
        \begin{aligned}
                & - \int_{x_{i - \frac{1}{2}}}^{x_{i + \frac{1}{2}}} \int_{y_{i - \frac{1}{2}}}^{y_{i + \frac{1}{2}}} \nabla D_g(\textbf{r}) \nabla \delta \phi_g(\textbf{r}, \omega) dx dy \\
                &= - D_{g,j,i} \biggl( \frac{d}{dx} \delta \phi_{j,i + \frac{1}{2}}^- - \frac{d}{dx} \delta \phi_{j,i - \frac{1}{2}}^+ \biggr) \Delta y\\
                &= \frac{D_{g,j,i} \Delta y}{\Delta x} \biggl( -\frac{2 D_{g,j,i-1}}{D_{g,j,i-1} + D_{g,j,i}} \delta \phi_{g,j,i-1} + \biggl( \frac{2 D_{g,j,i+1}}{D_{g,j,i+1} + D_{g,j,i}} + \frac{2 D_{g,j,i-1}}{D_{g,j,i-1} + D_{g,j,i}} \biggr) \delta \phi_{g,j,i} \\
                &  -\frac{2 D_{g,j,i+1}}{D_{g,j,i+1} + D_{g,j,i}} \delta \phi_{g,j,i+1} \biggr)
        \end{aligned}
\end{equation}
Similar process is done for the y-direction.

The boundary conditions used in this cases are mainly vacuum boundary conditions. However, the code is able to solve vacuum boundary conditions, reflective boundary conditions, and zero flux boundary conditions. The vacuum boundary condition is defined as follow,
\begin{equation}
        j_- = \frac{\phi(\textbf{r}_b)}{2} - D(\textbf{r}_b) \nabla \phi(\textbf{r}_b) = 0
\end{equation}
The reflective boundary conditions is defined as follow,
\begin{equation}
        \begin{aligned}
                j_+  &= j_-\\
                \frac{\phi(\textbf{r}_b)}{2} + D(\textbf{r}_b) \nabla \phi(\textbf{r}_b) &= \frac{\phi(\textbf{r}_b)}{2} - D(\textbf{r}_b) \nabla \phi(\textbf{r}_b)
        \end{aligned}
\end{equation}
Then the zero flux boundary condition is defined as follows,
\begin{equation}
        \phi(\textbf{r}_b) = 0
\end{equation}

\section{Spatial Discretization for Hexagonal Geometries}
\label{sec:hexagonal_discretization}
For the hexagonal geometries, the system 

\section{Simulated Cases for Verification Purposes}
\label{sec:simulated_vandv}

\subsection{2D C3 Benchmark}
Forward, Noise, ZPRTF, comparison

The C3 reactor benchmark is a case for heterogeneous system. It is based on the C3 benchmark on deterministic transport calculations \cite{cavarecBenchmarkCalculationsPower1994} and it is perturbed by introducing a localized neutron noise source. The system configuration is given in Figure \ref{fig:C3_configuration}. It consists of two UO2 fuel assemblies (at North-West and South-East positions) and two MOX fuel assemblies (at North-East and South-West positions). The size of each fuel assembly is 21.42 cm x 21.42 cm. The dark blue squares in the illustration are guide tubes; the ones in the center of the fuel assemblies contain fission chambers. Reflective boundary conditions are imposed. The perturbation is a fluctuation of 5\% of the fast and thermal neutron capture cross-sections in the fuel cell (16,19) identified with a red square in Figure \ref{fig:C3_configuration}.
\begin{figure}[h]
        \centering
        \includegraphics[width=0.6\textwidth]{figures/C3_configuration.png}
        \caption{Configuration of C3 Benchmark (taken from \cite{mylonakisNeutronNoiseSimulations2019})}
        \label{fig:C3_configuration}
\end{figure}

For this simulation, CORE SIM+ results using finite difference and discrete ordinate S8 simulation results are available. Therefore, the results from simulation will be compared to the CORE SIM+ results and discrete ordinate results.

\subsection{2D BIBLIS Benchmark}
Forward, Noise, ZPRTF, comparison

\subsection{2D VVER-400 Benchmark}
Forward, Noise, ZPRTF, comparison

\subsection{3D Generic PWR}
Forward, Noise, ZPRTF, comparison

\subsection{3D VVER-400 Benchmark}
Forward, Noise, ZPRTF, comparison

\section{HTTR Benchmark}
\label{sec:httr_benchmark}
transport code \texttt{MCNP} is used to generate the test cases. The neutron transport code \texttt{MCNP} is used to generate the test cases. The neutron transport code \texttt{MCNP} is used to generate the test cases. The neutron transport code \texttt{MCNP} is used to generate the test cases. The neutron transport code \texttt{MCNP} is used to generate the test cases. The neutron transport code \texttt{MCNP} is used to generate the test cases. The neutron transport code \texttt{MCNP} is used to generate the test cases. The neutron transport code \texttt{MCNP} is used to generate the test cases.

\subsection{2D HTTR Simulations}
Forward, noise, ZPRTF, comparison

\subsection{3D HTTR Simulations}
Forward, noise, ZPRTF, comparison

\section{Conclusions}